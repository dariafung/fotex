\documentclass{article}
\usepackage{amsmath}
\usepackage{amsfonts}
\usepackage{amssymb}

\begin{document}

\title{The Fibonacci Sequence}
\author{}
\date{}

\maketitle

\section{Introduction}

The Fibonacci sequence is a series of numbers where each number is the sum of the two preceding ones, starting from 0 and 1. It is defined recursively as follows:

\begin{equation}
F_n = F_{n-1} + F_{n-2}
\end{equation}
where $F_0 = 0$ and $F_1 = 1$.

\section{The Sequence}

The first few terms of the Fibonacci sequence are

0, 1, 1, 2, 3, 5, 8, 13, 21, 34, 55, 89, 144

\section{Mathematical Properties}

The Fibonacci sequence has many interesting mathematical properties.  Here are a few:

\begin{itemize}
    \item \textbf{The Ratio of Consecutive Fibonacci Numbers:} The ratio of consecutive Fibonacci numbers approaches the golden ratio, denoted by $\phi$:
    \begin{equation}
        \lim_{n \to \infty} \frac{F_{n+1}}{F_n} = \phi = \frac{1 + \sqrt{5}}{2} \approx 1.618
    \end{equation}
    \item \textbf{Binet's Formula:}  The $n$-th Fibonacci number can be calculated directly using Binet's formula:
    \begin{equation}
        F_n = \frac{\phi^n - (-\phi)^{-n}}{\sqrt{5}}
    \end{equation}

\end{itemize}

\section{Applications}

The Fibonacci sequence appears in various fields, including:

\begin{itemize}
    \item \textbf{Mathematics:} Number theory, combinatorics.
    \item \textbf{Nature:}  The arrangement of leaves on a stem, the spirals in a sunflower head, the branching of trees.
    \item \textbf{Computer Science:}  Algorithms, data structures.
\end{itemize}

\section{Example Code (Python)}

Here's a simple Python code to generate the Fibonacci sequence:

\begin{verbatim}
def fibonacci(n):
    a, b = 0, 1
    for _ in range(n):
        print(a)
        a, b = b, a + b

fibonacci(10)
\end{verbatim}

\end{document}